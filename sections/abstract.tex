%! TEX root = ../main.tex
\documentclass[main]{subfiles}

\begin{document}
\chapter{序論}
近年長野市における路線バスの利用者数は減少傾向であったが,平成22年に長野市地域公共交通総合連携計画策定された後,路線バスの利用者数は回復傾向にある.\cite{naganokeikaku}
将来において路線バスの利用率が上昇した場合,現状のままではバスの利用者,
バス運行会社の両者の視点から見た最適な運航スケジュールではない可能性がある.
そこで,先行研究では現在の長野市の公共交通システムのモデリングを行い,
バス利用率を変化させたモデルごとに,移動者の移動時間短縮とおよびバスの乗車率向上を目的として,
路線バス運行スケジュールの最適化を行った.\cite{senkoukenkyu}
結果,バス利用率を変化させた場合においても,利用率に合わせて適した運行スケジュールを得ることが出来た.
先行研究で明らかになった課題に,計算時間の長さが挙げられる.
先行研究では,長野駅を中心とした地域をモデルとした小規模なデータ設定であるにもかかわらず,計算時間に約一ヶ月ほど要した.
実際にバス運行スケジュール決定するにあたりそれほどの時間をかけて最適解を得ることはバス運行会社及び行政の立場から非現実的である.
そこで,本研究では計算時間の大半を占めるシミュレーションを高速な代理モデルを使うことで計算時間の高速化を狙う.
以下,第2章では路線バス運行スケジュール最適化の問題設定について説明する.
第3章では,先行研究の結果をまとめ,先行研究の課題点を述べる.
第4章では,代理モデルとして使うWalsh関数について説明し,第5章で代表的なテスト問題であるMNK問題についてWalsh関数を使い,
Walsh関数の代理モデルとしての能力を検証する.
第6章と第7章で,路線バス運行スケジュール最適化問題に代理モデルを使うために必要な実験を行い,第8章で実際に代理モデルを導入しその結果を述べる.
最後に,第9章で本論文をまとめ,今後の研究課題を述べる.
\end{document}