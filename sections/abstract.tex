%! TEX root = ../main.tex
\documentclass[main]{subfiles}

\begin{document}
\chapter{序論}
近年長野市における路線バスの利用者数は減少傾向であったが,平成22年に長野市地域公共交通総合連携計画策定された後,路線バスの利用者数は回復傾向である\cite{naganokeikaku}
将来において路線バスの利用率が上昇した場合,現状のバスの運航スケジュールのままではバスの利用者,
バス運行会社の両者の視点から見た最適な運航スケジュールではない可能性がある.
そこで,先行研究では現在の長野市の公共交通システムのモデリングを行い,
バス利用率を変化させたモデルごとに,移動者の移動時間短縮とおよびバスの乗車率向上を目的として,
路線バス運行スケジュールの最適化を行った.
結果,バス利用率を変化させた場合,路線によってはバスの発車感覚が短くなる傾向があるが,逆に長くなる路線も存在することが分かった.
先行研究で明らかになった課題について,計算速度が挙げられる.
先行研究では,長野駅を中心とした地域をモデルとした小規模なデータ設定であるにもかかわらず,計算時間に約一ヶ月ほど要した.
実際にバス運行スケジュール決定するにあたりそれほどの時間をかけて最適解を取得することはバス運行会社及び行政の立場から非現実的である.
そこで,本研究では計算時間の大半を占めるシミュレーションを高速なsurrogate modelにより代用することで計算時間の高速化を狙う.
以下,第2章では~
\end{document}