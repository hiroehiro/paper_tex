%! TEX root = ../main.tex
\documentclass[main]{subfiles}

\begin{document}
\chapter{結論}
本研究では,運行スケジュールを交通流シミュレータであるSUMOに入力し,シミュレーションの結果得られる総移動時間と乗車率の値を,
代理モデルであるWalsh関数によって予測することによって,路線バス運行スケジュール最適化問題を高速に解くことを狙った.

まず,有名なテスト問題であるMNK問題の評価関数をWalsh関数で予測させることで,Walsh関数の代理モデルとしての予測精度を検証した.
結果,非常に高い精度で予測することが出来ることが分かった.また,OrderというWalsh関数の複雑さを表す値の違いによって,
大きく予測精度と計算時間に差が生まれることが分かった.次に運行スケジュールをWalsh関数に入力するために必要なビット変換方法について
5つの手法を提案し検証した.結果,手法によって予測精度に数十倍の差が生まれる事が分かり,変換手法を適切に選ぶことの重要性が分かった.
また,学習データ数によるWalsh関数の予測精度を検証し,代理モデルを路線バス運行スケジュール最適化問題に導入する時の必要なデータ数を決定した.
最後に,代理モデルを路線バス運行スケジュール最適化問題に導入し,高速化が達成されたか検証した.
2つの導入方法を提案したが,一部区間で昨年度を上回る結果を残したものの,全体を通してはどちらも先行研究を高速化するには至らなかった.

最後に,本研究の目的である路線バス運行スケジュール最適化問題の高速化を達成するために考えられる,改善点や今後の展望を述べる.
まず,代理モデルの精度をさらに向上させる必要がある.本研究では代理モデルの決定係数が0.6の段階でアルゴリズムに導入していたが
,この代理モデルの精度の悪さによって高速化がなされなかった可能性が最も高いと思われる.
代理モデルの精度を上げるために,Walsh関数のOrderを増やしてモデルの複雑さを高める,
Walsh関数だけでなく複数の予測モデルを使うといった方法が考えられる.
次に,予測モデルの目的変数を変更する方法が考えられる.本研究では2目的最適化であったため,それぞれの目的変数を予測する
Walsh関数を作成し,2つのWalsh関数の予測値を使って評価値としていた.
しかし,NSGAⅡの親集団選択に用いられる非優越ソートでは正確にそれぞれの目的変数の値を予測する必要はなく,解集団の支配関係を
求めることが出来ればよい.よって,目的変数を非優越ランクにすることで予測に使う代理モデルを1つにし,
より本質的な予測が出来る代理モデルにすることによって,高速化が達成できる可能性がある.
最後に,代理モデル導入フローチャートの改良がある.本研究では42世代目以降に代理モデルを導入していたが,これは定性的に決めた数字であり,
定量的な根拠があるわけではない.また,代理モデル内で進化型の代理モデルでの進化は10世代行っていたが,これも定性的に決めた数字であり,
定量的な根拠はない.よって,これらを定量的に評価し導入することで,より効率よく代理モデルが使用され,高速化できると考えられる.
また,遺伝的アルゴリズムの進化初期は解空間の広大な範囲から解が選ばれているが,進化終盤は解空間の最適解に近い範囲に集中して
解が選ばれている.
このように,進化の段階で得られる学習データの特徴は異なると考えられる.
このことから,例えば進化初期のデータで学習させた代理モデルは多様な解を予測することができ,
進化終盤のデータで学習させた代理モデルは最適解に近い狭い範囲を高い精度で予測することができるなど,
進化の段階で代理モデルを複数用意し学習させることで,異なる解空間の予測を得意とする代理モデルが作成できる可能性がある.
このような,得意な範囲の異なる代理モデルを適切なタイミングで導入するようなフローチャートに変更すれば,より高速化が進むと考えられる.
\end{document}