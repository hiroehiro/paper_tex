%! TEX root = ../main.tex
\documentclass[main]{subfiles}

\begin{document}
\chapter{walsh関数を用いた代理モデル}
    前章で,先行研究の課題に計算時間の長さが上げられた.
    本研究では高速な代理モデルを使い,SUMOによるシミュレートを代理することによって計算時間の短縮を図る.
    本研究で代理モデルとして用いるWalsh関数について述べる.

    \section{Wlash関数}   
    Walsh関数\cite{walsh}は1923年にJ.L.Walshにより提唱された.
    Walsh関数はn個の入力$x \in \{0, 1\}, x_1, x_2, ..., x_n$の時,\ref{walsh_siki}で表される.
    \begin{equation}
        \begin{split}
            f(x_1, x_2, ..., x_n) &= w_1(-1)^{x_1} + w_2(-1)^{x_2} + ... + w_n(-1)^{x_n} \\
            &+ w_{12}(-1)^{x_1+x_2} + ... + w_{n-1n}(-1)^{x_{n-1}+x_n} \\
            &+ w_{123}(-1)^{x_1+x_2+x_3} + ... + w_{n-2n-1n}(-1)^{x_{n-2}+x_{n-1}+x_n} \\
            &+ ...
            \label{walsh_siki}
        \end{split}
    \end{equation}
    
    


\end{document}