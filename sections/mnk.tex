%! TEX root = ../main.tex
\documentclass[main]{subfiles}

\begin{document}
\chapter{実験1:MNK問題におけるWalsh関数の予測}
    Walsh関数の代理モデルとしての予測能力を測るために,最適化問題として有名なMNK問題について,Walsh関数を用いて予測する.
    \section{MNK問題}
        \subsection{概要}
        MNK問題\cite{mnk}は各目的の評価関数を最大化する問題である.
        変数間には相関があり,相関のある変数の数を増やすことにより,問題を複雑化することができる.
        MNK問題は,目的数M,設計変数数N,相関のある変数数Kの3つの要素で構成される.
        それぞれの値を変えることによって,問題の複雑度を設定することが出来る.

        \subsection{評価関数}
        \begin{table}[h]
            \centering
            \caption{評価値の例(M=2, N=6, K=1)}
            \begin{tabular}{cc|cc}
              $x_1$ & $x_2$ & $f_1(x_1, x_2)$ & $f_2(x_1, x_2)$\\ \hline
              0 & 0 & 1 & 5 \\
              0 & 1 & 3 & 2 \\
              1 & 0 & 6 & 2 \\
              1 & 1 & 3 & 4 \\
            \end{tabular}
            \label{mnk_table}
        \end{table}
        表\ref{mnk_table}に,M=2,N=2,K=1の時の設計変数と評価値の例を示す.
        $x_1$,$x_2$は変数であり,$f_1(x_1, x_2)$,$f_2(x_2, x_1)$は評価関数である.
        K=1,つまり相関のある変数数が1つなので,自身と相関のある変数が1つある.
        そして,その2つの変数に対して評価値が定義されている.
        M=2であるため,評価関数が2つ設定されている.
        N=6であるため,設計変数は6つ,例えば0,1,0,1,0,1などになる.
        設計変数の1番目と2番目の値を評価関数に入れ評価値を取得する.
        さらに,設計変数の2番目と3番目の値を評価関数に入れ評価値を取得する.
        これを設計変数の5番目と6番目まで繰り返し,評価値の総和を最終的な評価値とする.
    
        \section{実験方法}
        M=2,N=20,K=2の時のMNK問題に対して,まずランダムに生成した入力を20000件用意し,その評価値を求める.
        この20000件のデータの中,ランダムに18000件を選び学習データに,残りの2000件をテストデータにする.
        学習データを用いてWalsh関数を学習させ,パラメータ$w$を決定する.
        学習済みのWalsh関数を用いてテストデータの評価値を予測する.
        実際の正しいテストデータの評価値と比べ,Walsh関数の予測精度を見る.
        予測精度を測る評価指標には決定係数とMAEを用いる.
        また,Walsh関数にはOrder2のものとOrder3のものを使い,Orderの違いによる予測精度の差を確認する.

        \section{結果}
        結果を表\ref{mnk_result}に示す.
        \begin{table}[h]
            \centering
            \caption{MNK問題におけるWalsh関数の予測精度}
            \begin{tabular}{c|ccc}
              Order & 計算時間(秒) & 決定係数 & MAE\\ \hline
              2 & $19$ & $0.662$ & $0.0232$ \\
              3 & $128$ & $0.999$ & $2.575 \times 10^{-6}$ \\
            \end{tabular}
            \label{mnk_result}
        \end{table}






\end{document}
