%! TEX root = ../main.tex
\documentclass[main]{subfiles}

\begin{document}
\chapter{MNK問題におけるWalsh関数の予測}
    Walsh関数の代理モデルとしての予測能力を測るために,最適化問題として有名なMNK問題について,Walsh関数を用いて予測する.
    \section{MNK問題}
        \subsection{概要}
        MNK問題\cite{mnk}は各目的の評価関数を最大化する問題である.
        変数間には相関があり,相関のある変数の数を増やすことにより,問題を複雑化することができる.
        MNK問題は,目的数M,設計変数数N,相関のある変数数Kの3つの要素で構成される.
        それぞれの値を変えることによって,問題の複雑度を設定することが出来る.
        \subsection{評価関数}
    


\end{document}
